    

\documentclass[a4paper,12pt]{article}
\usepackage[english]{babel}
\usepackage{amsmath}
\usepackage{a4wide}

\usepackage{amsfonts}
\usepackage{amssymb}
\usepackage{graphicx}
\usepackage{color}
\usepackage{caption}
\usepackage{array}
\usepackage{pdfpages}
\usepackage{float}
\usepackage[round]{natbib}
\usepackage{multirow}
\usepackage{multicol}
\usepackage{amsxtra}
\usepackage{amsbsy}
\usepackage{bm}
\usepackage{accents}
\usepackage{chngcntr}
\usepackage{dcolumn}
\usepackage[none]{hyphenat}
\usepackage[affil-it]{authblk}
\usepackage{datetime}
\usepackage{colortbl}
\usepackage{footnote}
\makesavenoteenv{tabular}
\usepackage[flushleft]{threeparttable}
\usepackage[hyphens]{url}
\usepackage{placeins}
\usepackage{dcolumn}
\usepackage{longtable}
\usepackage{booktabs}
\usepackage{setspace}
\usepackage{changepage}  


\usepackage{amsfonts}
\usepackage{graphicx}
\usepackage{color}
\usepackage{caption}
\usepackage{array}
\usepackage{pdfpages}
\usepackage{float}
\usepackage[round]{natbib}
\usepackage{multirow}
\usepackage{multicol}
\usepackage{amsxtra}
\usepackage{amsbsy}
\usepackage{accents}
\usepackage{chngcntr}
\usepackage{tabularx}
\usepackage{dcolumn}
\usepackage[none]{hyphenat}
\usepackage[affil-it]{authblk}
\usepackage{datetime}
\usepackage[labelfont=bf]{caption}
\usepackage{titlesec}
\usepackage{endnotes}
\usepackage{csquotes}
\usepackage{epstopdf}





\date{\normalsize{September 2017}}
\title{\Large \bf Climate Change Awareness and Willingness to Pay for its Mitigation: \\ Evidence from the UK}
\author{Monika Novackova and Richard Tol}
\affil{\small{Department of Economics, University of Sussex, Falmer, UK}}


\parindent 0pt
\parskip 0.5em
\newcommand\starred[1]{\accentset{~~~~~\star}{#1}}


\newcounter{magicrownumbers}
\newcommand\rownumber{\stepcounter{magicrownumbers}\arabic{magicrownumbers}}

\begin{document}
\sloppy
\section*{Definition of drought}
 According to international meteorological community, drought can be defined as \textit{ 'prolonged absence or marked deficiency of precipitation'}, a \textit{'deficiency of precipitation that results in water shortage for some activity or for some group'} or a \textit{'period of abnormally dry weather sufficiently prolonged for the lack of precipitation to cause a serious hydrological imbalance'} (\citealp{IPCCtrenberth, Heim2002}). The International Panel for Climate Change recognise three types of droughts: \textit{(i)}~'Agricultural drought' which is defined in terms of moister deficits in upper layer of soil up to about one meter depth~\textit{(ii)} 'meteorological drought' which refers to prolonged deficit of precipitation and~\textit{(iii)} 'hydrological drought' which relates to low streamflow, lake and levels of groundwater (IPCC; \citealp{IPCCtrenberth, Heim2002}). 
 
 \cite{Trenberth2014} discuss different ways of formulation of the Palmer Drought Severity Index (\citealt{Sheffield2012, Dai2011}).  \cite{Sheffield2012} argue that way of formulation of PDSI can have substantial effect on estimation of changes in severity of droughts over time. According to other studies, differences in formulation of the PDSI do not play a big role and it is more important to formulate the index such that the required data are available and reliable (\citealp{Trenberth2014,vanderSchrier2011,Wang2012}). \cite{Trenberth2014} attribute the differences in results of  \cite{Sheffield2012} and \cite{Trenberth2014} mostly to disparities among various rainfall datasets and different baseline periods rather than different formulations of the PDSI.   

Drought can be measured in absolute terms (e.g. lake levels or amount of soil moisture) or in relative measures, which is for example the  PDSI \citep{Trenberth2014}. Because drought is defined based on one tail of probability distribution function of a drought measure, small decrease in mean can appear as substantial increase in frequency of droughts. This has caused confusions and therefore usage of percentiles of soil moisture or streamflow is recommended as a better measure than mean \citep{Trenberth2014}.



%\textcolor{green}{Precist HEIM A Review of Twentieth Century Drought Indices Used in the United States 2002}
\subsubsection*{Extreme events, disaster and hazards}
\cite{IPCC2012ch1} define extreme events as 'the occurrence of a value of a weather or climate variable above (or below) a threshold value near the upper (or lower) ends of the range of observed values of the variable'. Some authors define extreme events only in terms of meteorological phenomena (\citealp{easterling2000, Jentsch2007}), others include also consequential physiological impacts or other effects on humans and ecosystems (\citealp{IPCC2012ch1, young2002}).


According to \cite{IPCC2012ch1}, disasters are defined as 'severe alternations in the normal functioning of community or a society due to hazardous physical events interacting with vulnerable social conditions, leading to widespread adverse human, material, economic or environmental effects that require immediate emergency response to satisfy critical human needs and that may acquire external support for recovery.' The hazardous physical events may be of natural, socio-natural, or purely anthropogenic origin (\citealp{IPCC2012ch1, wisner2004risk}).

Hazard can be defined as 'the potential occurrence of a natural or human induced
physical event that may cause loss of life, injury, or other
health impacts, as well as damage and loss to property, infrastructure,
livelihoods, service provision, and environmental resources' \citet{IPCC2012ch1}.

 \section*{Economic Effects of Droughts}
 
Shifts in staple food demand curve are usually not very large. Hence, when staple food becomes scarce, its price is usually subject to a massive increase (\citealp{Brown2015, brown2014book}). For low income groups, this often leads to reduction in calorie intake, malnutrition and increased risk of related health problems (\citealp{Golden2011, Handa2006}). Local food prices are therefore a good indicator of food scarcity and insecurity (\citealp{baffes2017, Brown2015}). \cite{Brown2015} investigate effects of weather disturbances and international price changes on local food prices which serve as a proxy for food scarcity. They use Kalman Filter approach (see \cite{KoopmanSJ} for more details) and they focus on regions which contain large segments of low income population including locations in Africa, South Asia and Latin America. They conclude that almost $20\%$ of local market prices are affected by domestic weather disturbances, $9\%$ of them are affected by international price change and $4\%$by both of them. Based on whether or not international food price and weather shocks are significant in explaining local food prices, \cite{brown2014book} groups food markets in selected developing countries into four categories as follows: significantly affected by both international food prices and weather, significantly affected by weather but not international food prices, significantly affected by international food prices but not by weather and not significantly affected by either of them. \cite{brown2014book} then discus common characteristics of markets in each of these groups.

\cite{Ochieng2016} estimate effects of climate variability and change on agricultural production \footnote{Mesured as value of yields per acre in farm household} using panel data in Kenya. According to their results the effects are significant, yet different for different crops. Temperature has positive effect on tea and negative effect on production of maize and crop. Further, rainfall affects production of tea negatively. Another study, which finds positive correlation between precipitation and agricultural productivity is \cite{Vrieling2011}.





%\textcolor{blue}{look at many references in \cite{Brown2015} and \cite{Ochieng2016} !!!maybe also look if good references in \cite{OxfamIDS}??}


\cite{Pedram2016} estimate national production losses per disaster worldwide during ${1964-2007}$ using a statistical method called superposed epoch analysis. Besides drought, they focus on extreme heat, cold and flood events. They conclude that on average $10.1\%$ reduction of cereal production can be linked to droughts and $9.1\%$ reduction is attributable to extreme heat. They did not find any significant effect of extreme cold and floods on production. \cite{Mehrabi2017} estimate cumulative crop production losses resulting from heat and drought disasters over the same time period (${1964-2007}$). Their estimates are almost half of those of \cite{Pedram2016}. The biggest losses are in Botswana, Paraguay, Nigeria, Angola and USA.

\cite{OxfamIDS} uses the GLOBE Computable General Equilibrium model of the global economy to estimate food prices for various $2030$ scenarios. According to his results, climate change will lead to substantial increase in both domestic and world market crop prices in comparison to baseline scenario in the absence of climate change. However, the increase in prices can be substantially mitigated if appropriate adaptation measures will be taken in sub-Saharan Africa \citep{OxfamIDS}.

Large literature uses Ricardian approach to estimate impacts of climate change, including change in precipitation on crop revenue (\citealp{RicardianBello,kabubo2007, KMendelsohn2008, SeoMendelsohn}.

\section*{Trends}

 Strong downwards trend in precipitation has been observed in the tropics from $10^\circ$N to $10^\circ$S, especially after $1976/1977$ \citep{IPCCtrenberth}. During the period $1900-2005$, the climate has become wetter in many parts of the world (eastern parts of America, northern Europe, northern and central Asia) but it has became much drier in Mediterranean, Sahel, southern Africa and parts of Southern Asia. Furthermore, increased frequency of heavy rain events has been observed also in the areas with decline in total rainfall \citep{IPCCtrenberth}. \cite{Trenberth2014} argue that as a consequence of global warming, dry areas have strong tendency to get drier while wet areas are getting wetter. 
 
\bibliographystyle{apa}
\bibliography{referencesFS}
\end{document}