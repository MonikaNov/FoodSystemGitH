   

\documentclass[a4paper,12pt]{article}
\usepackage[english]{babel}
\usepackage{amsmath}
\usepackage{a4wide}

\usepackage{amsfonts}
\usepackage{amssymb}
\usepackage{graphicx}
\usepackage{color}
\usepackage{caption}
\usepackage{array}
\usepackage{pdfpages}
\usepackage{float}
\usepackage[round]{natbib}
\usepackage{multirow}
\usepackage{multicol}
\usepackage{amsxtra}
\usepackage{amsbsy}
\usepackage{bm}
\usepackage{accents}
\usepackage{chngcntr}
\usepackage{dcolumn}
\usepackage[none]{hyphenat}
\usepackage[affil-it]{authblk}
\usepackage{datetime}
\usepackage{colortbl}
\usepackage{footnote}
\makesavenoteenv{tabular}
\usepackage[flushleft]{threeparttable}
\usepackage[hyphens]{url}
\usepackage{placeins}
\usepackage{dcolumn}
\usepackage{longtable}
\usepackage{booktabs}
\usepackage{setspace}
\usepackage{changepage}  


\usepackage{amsfonts}
\usepackage{graphicx}
\usepackage{color}
\usepackage{caption}
\usepackage{array}
\usepackage{pdfpages}
\usepackage{float}
\usepackage[round]{natbib}
\usepackage{multirow}
\usepackage{multicol}
\usepackage{amsxtra}
\usepackage{amsbsy}
\usepackage{accents}
\usepackage{chngcntr}
\usepackage{tabularx}
\usepackage{dcolumn}
\usepackage[none]{hyphenat}
\usepackage[affil-it]{authblk}
\usepackage{datetime}
\usepackage[labelfont=bf]{caption}
\usepackage{titlesec}
\usepackage{endnotes}
\usepackage{csquotes}
\usepackage{epstopdf}





\date{\normalsize{September 2017}}
\title{\Large \bf Climate Change Awareness and Willingness to Pay for its Mitigation: \\ Evidence from the UK}
\author{Monika Novackova and Richard Tol}
\affil{\small{Department of Economics, University of Sussex, Falmer, UK}}


\parindent 0pt
\parskip 0.5em
\newcommand\starred[1]{\accentset{~~~~~\star}{#1}}


\newcounter{magicrownumbers}
\newcommand\rownumber{\stepcounter{magicrownumbers}\arabic{magicrownumbers}}

\begin{document}
\sloppy
\section*{What is drought?}
How we define drought is very important. Based on specification of drought, we can measure changes in aridity over time. According to international meteorological community, drought can be defined as \textit{ 'prolonged absence or marked deficiency of precipitation'}, a \textit{'deficiency of precipitation that results in water shortage for some activity or for some group'} or a \textit{'period of abnormally dry weather sufficiently prolonged for the lack of precipitation to cause a serious hydrological imbalance'} (\citealp{IPCCtrenberth, Heim2002}). The International Panel for Climate Change discuss three types of droughts: \textit{(i)}~'Agricultural drought' which is defined in terms of moister deficits in upper layer of soil up to about one meter depth~\textit{(ii)} 'meteorological drought' which refers to prolonged deficit of precipitation and~\textit{(iii)} 'hydrological drought' which relates to low streamflow, lake and levels of groundwater (IPCC; \citealp{IPCCtrenberth, Heim2002}). \cite{Trenberth2014} discuss definitions and measures of drought and their relation to contradictory results of two recent studies, in particular \cite{Sheffield2012} and \cite{Dai2011}. \cite{Sheffield2012} argue that drought has not increased much since $1960$ althought incorrect versions of Palmer Drought Severity Index (PDSI) give substantially different results. On the other hand, \cite{Dai2011} conclude that results differ only slightly for different forms of PDSI and all its forms indicates widespread drying. Besides difference in way of calculating the drought index, \cite{Trenberth2014} attribute the contradicting results to disparities among various rainfall datasets and different baseline periods.   

As discussed by \cite{Trenberth2014}, drought can be measured in absolute terms (e.g. lake levels or amount of soil moisture) or using relative measures, such as PDSI. Because drought is defined based on one tail of probability distribution function of a drought measure, small decrease in mean can appear as very big increase in frequency of droughts. This has caused confusions and therefore usage of percentiles of soil moisture or streamflow is recommended as a better measure than mean \citep{Trenberth2014}.

Specifying a reliable index which could be used as a basis for definition of drought seems to problematic. The degree of drought does not only depend on precipitation, but also on whether and how fast the moisture is carried away (so the index should also incorporate evapotranspiration, which PDSI does. It also accounts for balance of precipitation.) Thus, besides precipitation, the index should incorporate humidity, wind, solar and long-wave radiation data \citep{vanderSchrier2011}.  However, availability of reliable data for solar radiation is a real problem \citep{Wang2012}.

\subsection*{How do authors define drought in terms of distribution of index?}
\begin{itemize}

\item \cite{Trenberth2014}:
\end{itemize}`

\section*{trends..}

 Strong downwards trend in precipitation has been observed in the tropics from $10^\circ$N to $10^\circ$S, especially after $1976/1977$ \citep{IPCCtrenberth}. During the period $1900-2005$, the climate has become wetter in many parts of the world (eastern parts of America, northern Europe, northern and central Asia) but it has became much drier in Mediterranean, Sahel, southern Africa and parts of Southern Asia. Furthermore, increased frequency of heavy rain events has been observed also in the areas with decline in total rainfall \citep{IPCCtrenberth}. \cite{Trenberth2014} argue that as a consequence of global warming, dry areas have strong tendency to get drier while wet areas are getting wetter.
 
 \section*{Economic Effects of Droughts}
 
\cite{Pedram2016} estimate national production losses per disaster worldwide during $1964-2007$. Besides drought, they focus on extreme heat, cold and flood events. They conclude that on average $10.1\%$ reduction of cereal production can be linked to droughts and $9.1\%$ reduction is attributable to extreme heat. They did not find any significant effect of extreme cold and floods on production.  
\bibliographystyle{apa}
\bibliography{referencesFS}
\end{document}